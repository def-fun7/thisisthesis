\appendix
\chapter*{Appendix A: Derivations}
\addcontentsline{toc}{chapter}{Appendices}

\subsection{Mass of the Black hole}

According to \emph{Newton's 2nd law},
\begin{equation} \label{eq: 1.1}
F = ma \tag{1.1}
\end{equation}
In case, of acceleration due to gravity \((g)\), this can be written as, 
\begin{equation} \label{eq: 1.2}
F_g = mg \tag{1.2}
\end{equation}
And in accordance with \emph{Newton's Law of Gravitation}, 
\begin{equation} \label{eq: 1.3}
F_g = \frac{GMm}{r^2} \tag{1.3}
\end{equation}
From equation \eqref{eq: 1.2} and \eqref{eq: 1.3}, we get \(g\), Gravitational acceleration as, 
\begin{equation} \label{eq: 1.4}
g = \frac{GM}{r^2} \tag{1.4}
\end{equation}
Tidal acceleration \((a_t)\) or Gravitational gradient \((\Delta g)\) is defined as the rate of change of gravitational acceleration across small distance \(\Delta r\), given by the derivative of \(g\) from equation \eqref{eq: 1.4}, as
\begin{align} \label{eq: 1.8}
    \frac{dg}{dr} &= \frac{d}{dr} (\frac{GM}{r^2}) \tag{1.5}\\
    \frac{dg}{dr} &= GM\frac{d}{dr} (r^{-2}) \tag{1.6}\\
    \frac{dg}{dr} &= GM (2 r^{-3}) \tag{1.7}\\
    \frac{dg}{dr} &= \frac{2GM}{r^{3}} \tag{1.8}
\end{align}

For a finite change in r as \(\Delta r\), for a finite change in g, \(\Delta g\),
\begin{equation} \label{eq: 1.9}
\Delta g = \frac{dg}{dr} \Delta r \tag{1.9}
\end{equation}
from equation \eqref{eq: 1.9} and \eqref{eq: 1.8}, and replacing \(Delta r\) with \(h\) to represent height of a person, we get the same equation as in \eqref{eq:tidal acceleration}, 
\begin{equation} \label{eq: 1.10}
\Delta g = \frac{2GMh}{r^{3}} \tag{1.10}
\end{equation}
Till here, the derivation can be found in text books. Afterwards, we are gonna solve it for our purposes. 

We would like to derive an equation that calculate the mass of the BH \(M\), that has a certain \(\Delta g\) at the surface of it's event horizon, i.e.~at \(r_s\). 

As, \(r_s = 2GM/c^2 \) and putting \(r = r_s\) in equation \eqref{eq: 1.10}, we get, 
\begin{align} \label{eq: 1.16}
\Delta g &= 2GMh (\frac{c^2}{2GM})^3 \tag{1.11} \\
\Delta g &= 2GMh \frac{c^8}{8G^3 M^3} \tag{1.12} \\
\Delta g &= h \frac{c^8}{4G^2 M^2} \tag{1.13} \\
    M^2 &=  \frac{hc^8}{4G^2 \Delta g} \tag{1.14} \\
    M &= \sqrt{\frac{hc^8}{4G^2 \Delta g} \tag{1.15}} \\
    M &= \frac{c^3}{2G}\sqrt{\frac{h}{\Delta g} \tag{1.16}}  
\end{align}
where, \eqref{eq: 1.16} is the desired equation \eqref{eq: mass_at}.

\subsection*{Proper and Coordinate Distances}

The \emph{Schwarzchild Metric} is given by, 
\begin{equation} \label{eq: Sm}
ds^2 = -c^2d\tau^2 = -(1 - \frac{r_s}{r}) c^2 dt^2 + (1 - \frac{r_s}{r})^{-1} dr^2 + r^2 (d\theta^2 - sin^2\theta d\phi^2) \tag{2.1}
\end{equation}
where \((t, r, \theta, \phi)\) are \emph{Schwarzchild coordinates} that describe the curvature around a non-rotating, uncharged spherical body (black hole). 
for simplification of the equation to get the relation for proper and coordinate distance, we will assume a spacetime event that is fixed in time, \(dt = 0\), and no angular motion, i.e. a pure radial movement, hence, \(d\theta = d\phi = 0\), which simplifies equation \eqref{eq: Sm} to, 
\begin{equation} \label{eq: 2.2}
ds^2 = (1 - \frac{r_s}{r})^{-1} dr^2 \tag{2.2}
\end{equation}
then for a proper distance \((\Delta r')\) for coordinate distance \((\Delta r)\), we have, 
\begin{align*} \label{eq:2.3}
\Delta r' &= \sqrt{ds^2}  \\
\Delta r' &= \sqrt{(1- \frac{r_s}{r})^{-1} \Delta r^2}  \\
\Delta r' &= \Delta r \sqrt{(1 - \frac{r_s}{r})^{-1}} \\
\Delta r' &= \Delta r \frac{1}{\sqrt{1- \frac{r_s}{r}}} \tag{2.3} 
\end{align*}
where , substituting \(\sqrt{1 - (rs/r)}\) as \(q\) will gives us equation \eqref{eq: qr}. 

Now, Further more, 
\begin{align*} \label{eq: 2.4}
\Delta r &= \Delta r' \sqrt{1- \frac{r_s}{r}} \\
\Delta r &= \Delta r' \sqrt{\frac{r - r_s}{r}} \\
\Delta r &= \Delta r' \sqrt{\frac{r_s + \Delta r - r_s}{ r_s + \Delta r}} \text{    from eq \eqref{eq: r}} \\
\Delta r &= \Delta r' \sqrt{\frac{ \Delta r }{ r_s + \Delta r}} \\
\Delta r^2 &= \Delta r'^2 \frac{\Delta r }{ r_s + \Delta r} \\
0 &= \Delta r^2 (r_s + \Delta r) - \Delta r'^{2} \Delta r \\
0 &= \Delta r^3 + r_s\Delta r^2 - \Delta r'^{2} \Delta r \\
0 &= \Delta r(\Delta r^2 + r_s\Delta r - \Delta r'^{2})  \tag{2.4}
\end{align*}
from equation \eqref{eq: 2.4}, we get \(\Delta r = 0\) and the quadratic equation, 
\begin{equation*} \label{eq: 2.5}
\Delta r^2 + r_s\Delta r - \Delta r'^{2} = 0 \tag{2.5}\
\end{equation*}
comparing with the general quadratic equation, \(ax^2 + bx + c = 0\), we get, 
\begin{align*}
a &= 1 \\
b &= r_s \\ 
c &= -\Delta r'
\end{align*}
using quadratic formula \eqref{eq: QF}, 
\begin{equation} \label{eq: QF}
x = \frac{-b \pm \sqrt{b^2 - 4ac}}{2a} \tag{QF}
\end{equation}
putting values we get, 
\begin{equation} \label{eq: dr}
\Delta r = \frac{-r_s \pm \sqrt{r_s^2 + 4(\Delta r')r_s}}{2} \tag{2.6}
\end{equation}

\subsection*{Proper and coordinate Time}

For a timelike spacetime event, we will assume an event that is stationary in space, i.e. \(ds = d\theta = d\phi = 0\), which reduces the \emph{Schwarzchild Metric} to,
\begin{equation} \label{eq: pt}
-c^2d\tau^{2} = - (1 - \frac{r_s}{r}) c^2 dt^2 \tag{2.7}
\end{equation}
simplifying it by cancelling out \(-c^2\) on both sides, and taking the square root and neglecting the derivative, leaves the following, 
\begin{equation} \label{eq: pt2}
\tau = t \cdot \sqrt{1 - \frac{r_s}{r}} \tag{2.8}
\end{equation}
or, using equation \eqref{eq: q}, we get the same equation as \eqref{eq: tau1}
\begin{equation} \label{eq:Tau2}
\tau = tq \tag{2.9}
\end{equation}

