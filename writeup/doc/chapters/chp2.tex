\chapter{Review of Related Literature}\label{sec:review-of-related-literature}

For this chapter, we will briefly be going over the literature used in this work, and related work of this kind. The aforementioned literature does include books, 
articles, papers, internet publications and work related to or like ours include content from youtube videos to podcasts and review articles on the internet. As it 
is not in the realm of possibilities and sensibility to cover all of them, we will be mainly mentioning these most relevant and pertinent ones, which are as follows:

\begin{description}
    \item[Gravitation and Cosmology] book
    \item[Profound Physics] Website by \emph{Ville Hirvonen} about physics concepts including General Relativity
    \item[Physics Forums Insights] An article titled \emph{Learn time Dilation and Redshift for a Static Black Hole} by \emph{stevebd1}
    \item[ScienceClic English]  A youtube channel, about scientific concepts, including videos about falling into black holes as well as simulations of black hole
    \item[PBS SpaceTime] A youtube channel, currently hosted by \emph{Dr. Matt O'Dowd} about astrophysics, including playlist about black holes.
    \item[Code Pen] A Front end code online code editor, including many simulations of black hole like \emph{Black Hole (WEBGL shader)}.
    \item[Circular Bit] An online game using black hole physics and simulations.
\end{description}

For review purposes, instead of having dedicated sections for each of these sources we will simply have sections, namely, \emph{Mathematical Part}, \emph{Conceptual Parts} and \emph{Coding Part}.

\section{Mathematical Part}\label{sec:maths}

The most fundamental equation in our work, \emph{Schwarzschild Metric} (eq:~\ref{eq:s_eq}) was originally given by \textbf{Karl Schwarzschild} in 1916 paper 
titled \emph{`On the Gravitational field of a Mass point according Einstien's Theory'}~\cite{schwarzschild1999gravitationalfieldmasspoint} (1999 english translation reference). Since then, 
this metric has been used quite a good deal in research related to general relativity. In fact, it is safe to say Schwarzschild geometry is one of the most 
well studied and understood among all GR geometrics. 

For the same conditions as Schwarzschild (event horizon at \emph{2M}) in different coordinate geometrics like \emph{Edditngton-Finkelstein coordinates} as well as for in depth explanations, we refer to~\cite{hartle2003gravity} and~\cite{weinberg1972gravitation}. 
Here, we found in-depth mathematical as well as scientific base for our work along with world lines for the observers falling into the black hole. We did not however
used any of the advanced tensor mathematics or differential calculus for our work. 

Other main equations like for mass (eq~\ref{eq:lam4}) and tidal forces, were basic newtonian physics that we mostly dervied for ourselved but can easily be found 
in college physics books. The equations of proper and coordinate time and distance were derived from (eq:~\ref{eq:s_eq}) (see derivations.) and were also tallied 
over different internet sources like~\cite{dummiesgr_web} and~\cite{physicsforums_timedilation}. 

\section{Conceptual Parts}\label{sec:conceptual}

What happens near a black hole or inside it's event horizon is still an open mystery. In fact, till recently, the very existence of entities like black hole was
under research, and to top that the matter of whether or singularities exist is still a hot question~\cite{kerr2023blackholessingularities}. Keeping all this in 
sight, the prospect of how an exchange of information among two observers each at a different gravitational well and hence experiencing rate of flow of time
differently was no easy feat. 

The work of~\cite{hartle2003gravity} and~\cite{weinberg1972gravitation} provided the insight into the flow of time for the observers falling into a black hole, as well as 
time at certain distance \((r)\) from the center of the black hole. Along with equations and mathematical framework providing the quantative understanding 
of the scenario, the provided explanations and thought experiments were also very insightful. 

More than simple time dilation and length contraction or gravitational redshift is experienced at close proximity of the black hole. The matter of gravitational 
lensing, the view based on the geodesics of light coming in to the black hole from outside universe, the horizon and the look of the view, all of that was also explained 
in that texts as well as some rather nice youtube videos like one of \emph{ScienceClic English}~\cite{scienceclic_blackhole}, provides stunning visuals as well as the explanations 
for the fall into the black hole at some distance. 

Beside that the work of \emph{Andrew Hamilton} ~\cite{hamilton_insidebh} gives a nice insight into the dynamics of the black hole, (though sadly the \emph{Black Hole Flight Simulator (BHFS)} is not publically available).
For Schwarzschild geometrics, his simulations like~\cite{hamilton_schw} are really awesome and helpful. 

\section{Coding Part}\label{sec:code}

The last part of our 
